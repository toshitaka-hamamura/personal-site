\documentclass[a4paper]{article}
%----packages----%
\usepackage[margin=3cm]{geometry}
\usepackage{hanging}
\usepackage{url}
\usepackage{hyperref}
%\usepackage{xurl} % url over the page break
%\usepackage{pagecolor}
\usepackage{fancyhdr} % for header
%----preamble----%
\fancyhead{} % clear all header fields
\pagestyle{fancy} % for header
\fancyhead[RO,LE]{CV, T. Hamamura}
\begin{document}
\sloppy
\title{Curriculum Vitae}
\author{Toshitaka (Toshi) Hamamura}
\maketitle

\section{Contact}
\textbf{Address:}\\
College of Education, Psychology, \& Social Work, Flinders University\\
Sturt Rd, Bedford Park, South Australia 5042\\
\\
\textbf{E-mail address:}\\	
toshitaka.hamamura@flinders.edu.au
%hamamura@19.alumni.u-tokyo.ac.jp (Emails will be forwarded to my active account)

\section{Professional Position}
\begin{itemize}
	\item 2017/04---2019/03: \textbf{JSPS Research Fellow (DC2)}, Japan Society for the Promotion of Science (Host Institution: University of Tokyo)
	\item 2019/04---2020/03: \textbf{Associate Researcher}, KDDI Research, Inc.
	\item 2020/04---2023/03: \textbf{JSPS Research Fellow (PD)}, Japan Society for the Promotion of Science (Host Institution: National Center of Neurology and Psychiatry)
	\item 2023/04---present: \textbf{JSPS Overseas Research Fellow}, Japan Society for the Promotion of Science (Host Institution: Flinders University)
\end{itemize}

\section{Other Position and Affiliation}
\begin{itemize}
	\item 2020/04---present: \textbf{JSPS Research Fellow}, National Center for Cognitive Behavior Therapy and Research, National Center of Neurology and Psychiatry
	\item 2020/04---2024/03: \textbf{Invited Researcher}, KDDI Research, Inc.
	\item 2022/12---2023/01: \textbf{Visiting Scholar}, Department of Psychology, National Taiwan University
	\item 2023/04--present: \textbf{Visiting Scholar}, College of Education, Psychology \& Social Work, Flinders University
\end{itemize}

\section{Education}
\begin{itemize}
	\item 2006---2010: \textbf{Bachelor of Art in Psychology}, Trinity Western University, Langley, BC, Canada. Minor in Business Administration. \textit{Graduating with distinction.}
	\item 2012---2015: \textbf{Master of Science in Clinical Psychology}, California State University, Fullerton, CA, USA. 
	\item 2016---2019: \textbf{Doctor of Philosophy in Education (Clinical Psychology)}, University of Tokyo, Bunkyo, Tokyo, Japan. Minor in Social ICT Graduate Program.
\end{itemize}

\section{Licenses}
\begin{itemize}
	\item 2019---present: \textbf{Clinical Psychologist}, Foundation of the Japanese Certification Board for Clinical Psychologists. (Japan)
	\item 2019---present: \textbf{Certified Public Psychologist}, Ministry of Health, Labour, and Welfare / Ministry of Education, Culture, Sports, Science and Technology. (Japan)
\end{itemize} 

\section{Clinical Experience}
\begin{itemize}
	\item 2010---2012: \textbf{Residential Care Worker}, Developmental Disabilities Association, Richmond, BC, Canada
	\item 2011---2012: \textbf{Mental Health Worker}, Coast Mental Health, Vancouver, BC, Canada
	\item 2016---2017: \textbf{Mental Health Worker}, Taito City Health Center, Tokyo, Japan
	\item 2017---2018: \textbf{Staff Counselor}, Student Support Office, Graduate School of Science, University of Tokyo, Japan
	\item 2018---2019: \textbf{Staff Counselor}, Support Room, Graduate School of Economics, University of Tokyo, Japan
	\item 2019---2023: \textbf{Clinical Psychologist}, National Center for Cognitive Behavior Therapy and Research, National Center of Neurology and Psychiatry, Japan
	\item 2020---2023: \textbf{Clinical Psychologist}, Psychiatric Unit, Tokyo Medical and Dental University, Japan
	\item 2023---present: \textbf{Clinical Psychologist}, Private practice
\end{itemize}

\section{Teaching Experience}
\begin{itemize}
	\item 2013---2015: \textbf{Graduate assistant}, California State University, Fullerton, Fullerton, USA (Research methods; Learning and memory; Cognitive psychology; Comparative animal bahavior)
	\item 2017---2019: \textbf{Teaching assistant}, University of Tokyo, Tokyo, Japan (Intelligence Test)
	\item 2019---2022: \textbf{Part-time lecturer}, Seikei University, Tokyo, Japan (Introduction to Psychology)
	\item 2021---2022: \textbf{Part-time lecturer}, Tokyo Women's Christian University, Tokyo, Japan (Third-year undergraduate seminar)
\end{itemize}

\section{Publication}
\subsection{Peer-Reviewed Articles}
\begin{enumerate}
	\item \underline{Hamamura, T.}, \& Laird, P.G. (2014) The effect of perfectionism and acculturative stress on levels of depression experienced by East Asian international students. \textit{Journal of Multicultural Counseling Development}, 42, 205-217. \url{https://doi.org/10.1002/j.2161-1912.2014.00055.x}
	\item \underline{Hamamura, T.}, Suganuma, S., Ueda, M., Shimoyama, H., \& Mearns, J. (2018). Standalone effects of a cognitive behavioral intervention using a smartphone application on psychological distress and alcohol consumption among Japanese workers: A non-randomized controlled trial. \textit{JMIR Mental Health}, 5(1):e24. \url{http://doi.org/10.2196/mental.8984}
	\item \underline{Hamamura, T.}, Suganuma, S., Takano, A., Matsumoto, T., \& Shimoyama, H. (2018). The efficacy of a web-based screening and brief intervention for reducing alcohol consumption among Japanese problem drinkers: Protocol of a single-blind randomized controlled trial. \textit{JMIR Research Protocol}, 7(5):e10650.  \url{http://doi.org/10.2196/10650}
	\item \underline{Hamamura, T.}, \& Mearns, J. (2019). Depression and somatic symptoms in Japanese and American college students: Negative mood regulation expectancies as a personality correlate. \textit{International Journal of Psychology}, 54(3), 351-359. \url{http://doi.org/10.1002/ijop.12467}
	\item \underline{Hamamura, T.}, \& Mearns, J. (2020). Mood induction changes negative alcohol expectancies among Japanese adults with problematic drinking: Negative mood regulation expectancies moderate the effect. \textit{International Journal of Mental Health and Addiction}, 18, 195-205. \url{http://doi.org/10.1007/s11469-018-9991-8}
	\item Oka, T., \underline{Hamamura, T.}, Miyake, Y.,  Kobayashi, N., Honjo, M., Kawato, M., Kubo, T., \& Chiba, T. (2021). Prevalence and risk factors of internet gaming disorder and problematic internet use before and during the COVID-19 pandemic: A large online survey of Japanese adults. \textit{Journal of Psychiatric Research}, 142, 218-225. \url{https://doi.org/10.1016/j.jpsychires.2021.07.054}
	\item Oka, T., Kubo, T., Kobayashi, N., Nakai, F., Miyake, Y., \underline{Hamamura, T.}, Honjo, M., Toda, H., Boku, S., Kanazawa, T., Nagamine, M.,  Cortese, A., Takebayashi, M., Kawato, M., Chiba, T. (2021). Multiple time measurements of multidimensional psychiatric states from immediately before the COVID-19 pandemic to one year later: a longitudinal online survey of the Japanese population. \textit{Translational Psychiatry}, 11, 573. \url{https://doi.org/10.1038/s41398-021-01696-x}
	\item \underline{Hamamura, T.}, Suganuma, S., Takano, A., Matsumoto, T., \& Shimoyama, H. (2022). The effectiveness of a web-based intervention for Japanese adults with problem drinking: An online randomized controlled trial. \textit{Addictive Behaviors Reports}, 15, 100400. \url{https://doi.org/10.1016/j.abrep.2021.100400}
	\item An, T., \underline{Hamamura, T.}, Kishimoto, T., Mearns, J. (2022). Negative Mood Regulation Expectancies Moderate the Effects of Acculturative Stress on Affective Symptoms among Chinese International Students in Japan. \textit{Japanese Psychological Research}. \url{https://doi.org/10.1111/jpr.12408}
	\item \underline{Hamamura, T.}, Kobayashi, N., Oka, T., Kawashima, I., Sakai, Y., Tanaka, S. C., \& Honjo, M. (2023). Validity, reliability, and correlates of the Smartphone Addiction Scale, Short Version among Japanese adults. \textit{BMC Psychology}, 11, 78. \url{https://doi.org/10.1186/s40359-023-01095-5}
	\item \underline{Hamamura, T.}, Kurokawa, M., Mishima, K., Konishi, T., Nagata, M., \& Honjo, M. (2023). Standalone effects of focus mode and social comparison functions on problematic smartphone use among adolescents. \textit{Addictive Behaviors}, 147, 107834. \url{https://doi.org/10.1016/j.addbeh.2023.107834}
	\item Chiba, T., Ide, K., Murakami, M., Kobayashi, N., Oka, T., Nakai, F., Yorizawa, R., Miyake, Y., \underline{Hamamura, T.}, Honjo, M., Toda, H., Kanazawa, T., Boku, S., Kubo, T., Hishimoto, A., Kawato, M., \&  Cortese, A. (2023). Event-related PTSD symptoms as a high-risk factor for suicide--Longitudinal observational study. \textit{Nature Mental Health}, 1, 1013--1022. \url{https://doi.org/10.1038/s44220-023-00157-2}
	\item King, D., Nogueira-López, A., Galanis, C., \underline{Hamamura, T.}, Bäcklund, C., Giardina, A., Billieux, J., \& Delfabbro, P. (2023). Reconsidering item response categories in gaming disorder symptoms measurement. \textit{Journal of Behavioral Addictions}, 12(4), 873--877. \url{https://doi.org/10.1556/2006.2023.00070}
	\item Kobayashi, N., Jitoku, D., Mochimatsu, R., \underline{Hamamura, T.}, Honjo, M., Takagi, S., Sugihara, G. and Takahashi, H., 2024. Treatment readiness and prognosis for problematic smartphone use: Evaluation of the Stages of Change, Readiness, and Treatment Eagerness Scale (SOCRATES) and log data. \textit{Psychiatry and Clinical Neurosciences Reports}, 3(1), p.e172. \url{https://doi.org/10.1002/pcn5.172}
\end{enumerate}

\subsection{Non-Peer-Reviewed Articles and Preprints}
\begin{enumerate}
	\item Ueda, M., \underline{Hamamura, T.}, Nakamura, A., \& Shimoyama, H. (2017). The application of animal research on the gene-environment interaction: A review. \textit{Bulletin of the Division of Clinical Psychology}, University of Tokyo, 40, 22-29.
	\item Uchimura, Y., \underline{Hamamura, T.}, Kitahara, Y., Oka, M., Suzuki, T., Kobayashi, N., \& Shimoyama, H., (2018). A Review on the population approach using Acceptance and commitment therapy: Increase of well-being and prevention effects. \textit{Bulletin of the Division of Clinical Psychology}, University of Tokyo, 41, 34-41.
	\item Takaseki, H., \underline{Hamamura, T.}, Lee, J., \& Shimoyama, H. (2019). Current support system and challenges on isolated mothers with risk of abuse. \textit{Bulletin of the Division of Clinical Psychology}, University of Tokyo, 42.
	\item Quaranta, T., \underline{Hamamura, T.}, \& Mearns, J. (2023). Correlates of Authoritarian versus Authoritative Parenting in the West and in Japan: Is This Dichotomy Valid in Japan?. \textit{OSF Preprints}. \url{https://doi.org/10.31219/osf.io/g9ep7}
	\item Nakai, F., Kubo, T., Oka, T., Kobayashi, N., Tanichi, M., Murakami, M., \underline{Hamamura, T.}, Honjo, M., Miyake, Y., Ide, K., Cortese, A.. Pre-event psychiatric states predict trajectories of post-traumatic stress symptoms during the COVID-19 pandemic. \textit{medRxiv}. 2024:2024-01. \url{https://doi.org/10.1101/2024.01.14.23300571}
	\item \underline{Hamamura, T.}, Kaneko, K., \& Ito, M. (2024, March 29). Association between gaming disorder and internalizing symptoms among children and adolescents: A child-parent dyadic study. \textit{PsyArXiv.} \url{https://doi.org/10.31234/osf.io/72fb3}
	\item \underline{Hamamura, T.}, Oka, T., Sakai, Y., Tanaka, S. C., Chiba, T., Honjo, M., \& Kobayashi, N. (2024). Decreases in problematic smartphone use among adults in general after the COVID-19 outbreak: A three-year prospective study. \textit{OSF Preprints}. \url{https://doi.org/10.31219/osf.io/68gms}
\end{enumerate}
	 
\section{Academic Presentation}
\subsection{Invited Presentations}
\begin{enumerate}
	\item \underline{Hamamura, T.}, Suganuma, S., Takano, A., Matsumoto, T., \& Shimoyama, H. (2018, September 9--13). \textit{How effective is a brief website intervention with personalized normative feedback among Japanese adults with risky drinking? Findings from a pilot RCT.} In A. Takano \& T. Baba (Chair), Possibilities and challenges using e-health and m-health for addiction treatment. Symposium conducted at the meeting of at the 19th Congress of International Society for Biomedical Research on Alcoholism, Kyoto, Japan.
	\item \underline{Hamamura, T.} (2024, March 1). \textit{Cognitive-Behavioral Therapy for Treating Problematic Online Behaviors}. UMinn \& NTU Joint Symposium on Mental Health: Addiction and Traumatology.
\end{enumerate}
\subsection{Other Presentations}
\begin{enumerate}
	\item \underline{Hamamura, T.} (2010, June). \textit{The Role of Perfectionism, GPA Satisfaction, and Acculturation on Depression among International and Domestic Students} [Poster presentation]. Connecting Mind, an undergraduate research conference in psychology. Richmond, Canada.
	\item \underline{Hamamura, T.} \& Laird, P.G., (2013, June). \textit{Impact of Perfectionism and Acculturation on Levels of Depression Experienced by International Students} [Poster presentation]. International Association for Cross-Cultural Psychology, Los Angeles, U.S.A. (Poster)
	\item \underline{Hamamura, T.}, \& Mearns, J. (2016, June). \textit{Depression and somatic symptoms in Japanese and American college students: negative mood regulation expectancies as a personality predictor} [Poster presentation]. The 31st International Congress of Psychology, Yokohama.
	\item \underline{Hamamura, T.}, Mearns, J. (2018, September). \textit{Music mood induction alters negative alcohol expectancies among Japanese adults with problematic drinking: findings from an Internet Experiment} [Poster presentation]. The 41st Annual Research Society on Alcoholism Scientific Meeting, San Diego, U.S.A.
	\item \underline{Hamamura, T.} (2018, September). \textit{Relationships among expectancies, drinking motivation and problem drinking among Japanese adults: The role of expectancies for negative mood regulation} [Poster presentation]. The 19th Congress of International Society for Biomedical Research on Alcoholism, Kyoto, Japan.
	\item \underline{Hamamura, T.}, Kawai, K., Uchimura, Y., Suganuma, S., Sato, K., \& Shimoyama, H. (2019, March). \textit{Does a self-monitoring mobile app help reduction of problem drinking?: A pilot randomized controlled trial among Japanese problem drinkers} [Poster presentation]. International Congress of Psychological Science, Paris, France.
	\item \underline{Hamamura, T.}, Konishi, T., Kurokawa, M., Mishima K., T., \& Honjo, M. (2020, February). \textit{Development and evaluation of an Android application for appropriate smartphone use among Japanese adolescents} [Poster presentation]. 2020 Society for Personality and Social Psychology Annual Convention, New Orleans, U.S.A.
	\item Yasudomi, K., \underline{Hamamura, T.}, Honjo, M., Yoneyama, A., \& Uchida, M. (2021, March). \textit{Usage Prediction and Effectiveness Verification of App Restriction Function for Smartphone Addiction} [Paper presentation]. IEEE International Conference on E-health Networking, Application Services (HEALTHCOM), 1-8. doi: 10.1109/HEALTHCOM49281.2021.9398974.  \url{https://doi.org/10.1109/HEALTHCOM49281.2021.9398974}
	\item Yokoyama, C., Komazawa, A., Yahata, A., Miyamae, M., Kanie, A., \underline{Hamamura, T.}, \&, Ito, M. (2022, September). \textit{A Case Report of Positive Valence System-Focused Cognitive Behavioral Therapy Assisted by Virtual Reality for Postpartum Depression} [Poster presentation]. International Marc\'e Society for Perinatal Mental Health Biennial Meeting, London, U.K.
	\item \underline{Hamamura, T.}, Kobayashi, N., Oka, T., Chiba, T., \& Honjo, M. (2023, August 23--25). \textit{Has problematic smartphone use changed between before and three years into the pandemic?} [Conference session]. The 8th International Conference on Beahvioral Addictions, Incheon, Korea.
	\item Kaneko, K., \underline{Hamamura, T.}, Fujisato, H., Király, O., Demetrovics, Z. \& Ito, M. (2023, August 23--25). \textit{Development and Validation of the Japanese Version of the Motives for Online Gaming Questionnaire} [Poster presentation]. The 8th International Conference on Behavioral Addictions, Incheon, Korea.
	\item Furusawa, S., \underline{Hamamura, T.}, Yoneyama, A., Honjo, M., Kobayashi, N., Jitoku, D., Uchida, M. (2023, May 12--14). \textit{Visualization Methods to Support the Medical Treatment of the Problematic Use of Smartphones} [Oral presentation]. 2023 the 7th International Conference on Medical and Health Informatics (ICMHI), Kyoto, Japan.  \url{https://doi.org/10.1145/3608298.3608337}
\end{enumerate}

\subsection{Domestic Conference}
\begin{enumerate}
	\item \underline{Hamamura, T.}, Suganuma, S., Ueda, M., \& Shimoyama, H. (2017, September). The effect of self-monitoring application on drinking and psychological distress. Poster presented at the 39th Annual Convention of the Japanese Society of Alcohol-Related Problems, Yokohama, Japan.
	\item \underline{Hamamura, T.}, Nakamura, A., Yoshida, S., Mearns, J., \& Shimoyama, H. (2017, September). The Effect of virtual reality facial feedback on affect and autobiographical recall: A role of negative mood regulation expectancies. Poster presented at the 81st Annual Convention of the Japanese Psychological Association, Kurume, Japan.
	\item \underline{Hamamura, T.} (2018, September). The role of drinking quantity and alcohol expectancies as predictors of alcoholism tendency. Poster presented at the 82nd Annual Convention of the Japanese Psychological Association, Sendai, Japan.
	\item \underline{Hamamura, T.}, Honjo, M., Kurokawa, M., Mishima, K., Konishi, T., Nagata, M., \& Yoneyama, A. (2019, September). Do Internet addiction types predict smartphone use among adolescents?. Poster presented at the 83rd Annual Convention of the Japanese Psychological Association, Osaka, Japan.
	\item An, T., \underline{Hamamura, T.}, Kishimoto, T., \& Mearns, J. (2019, September). The Effects of School and Acculturative Stress on Depression and Anxiety Experienced by Chinese International Students in Japan: How Do Negative Mood Regulation Expectancies Moderate these Effects?. Poster presented at the 83rd Annual Convention of the Japanese Psychological Association, Osaka, Japan.
	\item \underline{Hamamura, T.}, Honjo, M., Kurokawa, M., Mishima, K., Konishi, T., Nagata, M., \& Yoneyama, A. (2019, October).The effects of the smartphone-based intervention on smartphone addiction among secondary school students: a randomized controlled trial using app-recorded measures. Oral presentation at the Japanese Alcohol, Nicotine, \& Drug Addiction Conference 2019, Hokkaido, Japan.
	\item Mishima, K., Kurokawa, M., Nagata, M., Konishi, T., \underline{Hamamura, T.}, Honjo, M., Yoneyama, A. (2020, March). The differences on smartphone addiction tendencies in the subjective measures on daily life and individual traits. [Oral presentation]. The Institute of Electronics, Information and Communication Engineers General Conference, online.
	\item \underline{Hamamura, T.}, Mearns, J. (2020, September). Drinking alone or with others? Excessive drinking among elders. Poster presented at the 84th Annual Convention of the Japanese Psychological Association, Tokyo, Japan.
	\item Kurokawa, M., Mishima, K., \underline{Hamamura, T.}, Konishi, T., Nagata, M., Honjo, M., Yoneyama, A. (2020, September8--November 2). Examination of the short version of the Internet addiction in smartphone use scale. Poster presented at the 84th Annual Convention of the Japanese Psychological Association, Tokyo, Japan.
	\item \underline{Hamamura, T.} (2020, September).  Definitions of smartphone dependence and its association with mental health.	In \underline{Hamamura, T.} (Chair). Current situations in smartphone dependence and appropriate interventions from the educational, medical, and industrial fields. Symposium conducted at the meeting of at the 84th Annual Convention of the Japanese Psychological Association, Tokyo, Japan.
	\item \underline{Hamamura, T.}, Kobayashi, N., Honjo, M., Miyake, Y., Chiba, T., Kawashima, I., Sakai, Y., Tanaka, S., Yoneyama, A. (2020, November). How is high smartphone dependence different from high smartphone use? Comparisons on personality and daily activities. Oral presentation at the Japanese Alcohol, Nicotine, \& Drug Addiction Conference 2020, held online.
	\item \underline{Hamamura, T.}, Kobayashi, N., Miyake, Y., Oka, T., Chiba, T., Honjo, M. Yoneyama, A. (2021, September). Has smartphone addiction changed during the pandemic?. Poster presented at the 84th Annual Convention of the Japanese Psychological Association, Tokyo, Japan.
	\item \underline{Hamamura, T.}, Kitamura, M., Asano, H., Honjo, M. Yoneyama, A. (2021, December). A digital intervention to prevent problematic smartphone use: Development and a pilot trial of a smartphone application for children and parents. Oral presentation at the Japanese Alcohol, Nicotine, \& Drug Addiction Conference 2021, held online.
	\item Kobayashi, N., Jitoku, D., Nakajima, R., \underline{Hamamura, T.}, Honjo, M., Sugihara, G., Takahashi, H. (2022, December). Relationships between problematic internet and gaming use and bone density in outpatients. Oral presentation at the Japanese Alcohol, Nicotine, \& Drug Addiction Conference 2022.
	\item \underline{Hamamura, T.}, Kobayashi, N., Jitoku, D., Nakajima, R., Takahashi, H., Honjo, M. (2022, September). Objective measurement of problematic smartphone use among A clinical sample. Oral presentation at the Japanese Alcohol, Nicotine, \& Drug Addiction Conference 2022.
	\item \underline{Hamamura, T.}, Kaneko, K., Ito, M. (2022, November 11). Associations of gaming disorder with internalizing and family functioning among children and adolescents. Oral presentation at the Japanese Association for Cognitive Therapy.
 	\item Ono, K., Tokushige, M., Hiratani, M., Kaneko, K., \underline{Hamamura, T.}, Miyamoto, Y., Tateno, M., Fujisato. H., Ito, M., Takano. A. (2023, October 13--15). \textit{Reliability and validity of a Japanese version of the Internet Gaming Disorder Scale for Children (IGDS-C)} [Oral presentation]. The Japanese Alcohol, Nicotine, \& Drug Addiction Conference 2023.
	\item Sugiyama, K., \underline{Hamamura, T.}, Meguro, T., Taya, M., Honjo, M., Kobayashi, N., Jitoku, D., Uchida, M.(2024, April 04--08). Evaluation of the validity of usage state estimation for smartphone dependency patients and improvements in practicality. [Oral presentation]. The Institute of Electronics, Information and Communication Engineers: General Conference, Hiroshima.
	\item Koyama, Y., \underline{Hamamura, T.}, Meguro, T.,\ Taya, M., Honjo, M., Kobayashi, N., Jitoku, D., Uchida, M.(2024, April 04--08). Analysis of smartphone usage habits focusing on battery remaining capacity. [Oral presentation]. The Institute of Electronics, Information and Communication Engineers: General Conference, Hiroshima.
	\item Hayashi, K., \underline{Hamamura, T.}, Meguro, T.,\ Taya, M., Honjo, M., Kobayashi, N., Jitoku, D., Uchida, M.(2024, April 04--08). Estimation of usage posture for smartphone dependency patients using K-means clustering. [Oral presentation]. The Institute of Electronics, Information and Communication Engineers: General Conference, Hiroshima.
	\item Kim, H., \underline{Hamamura, T.}, Meguro, T.,\ Taya, M., Honjo, M., Kobayashi, N., Jitoku, D., Uchida, M.(2024, April 04--08). Redefining Smartphone Apps Categories Based on Actual Usage. [Oral presentation]. The Institute of Electronics, Information and Communication Engineers General Conference, Hiroshima.
\end{enumerate}

\section{Other Publication}
	\begin{enumerate}
		\item \underline{Hamamura, T.} (2019). An article summary: Is smartphone addiction really an addiction? (Panova, T. \& Carbonell, X., 2018) Japanese Journal of Psychotherapy, p.602-603
		\item \underline{Hamamura, T.} (2020). Book review: Treatment for substance use disorder (Matsumoto, T., 2020). Japanese Journal of Psychotherapy, p.715
	\end{enumerate}

\section{Grant Activities}
\begin{description}
	\item 2016---2017: The University of Tokyo, Graduate Program for Social ICT Global Creative Leaders. Effects on Virtual Reality Facial Feedback on Affect and Autobiographical Memory. (Group Project; Principal Investigator). Grant amount: ¥1,500,000
	\item 2016---2017: The University of Tokyo, Graduate Program for Social ICT Global Creative Leaders. Effects of Self-Record Smartphone Application on Psychological Symptoms and Drinking Consumption. (Individual Project; Principal Investigator). Grant amount: ¥300,000
	\item 2017---2019: Grant-in-Aid for JSPS Fellows (DC2), Japan Society for the Promotion of Science. The Cultural and Developmental Role of Negative Mood Regulation Expectancies. (Individual Project; Principal Investigator). Grant amount: ¥2,100,000
	\item 2017---2018: The University of Tokyo, Graduate Program for Social ICT Global Creative Leaders. Understanding Mechanisms and Functionality of Self-Monitoring App and its Application in Clinical Settings. (Individual Project; Principal Investigator). Grant amount: ¥300,000
	\item 2017---2018: The University of Tokyo, Graduate Program for Social ICT Global Creative Leaders. Examining Effects of ICT-based Educational Program Using Feedback with Mobile Devices for Reducing Heavy Drinking. (Individual Project; Principal Investigator). Grant amount: ¥300,000
	\item 2018---2019: The University of Tokyo, Graduate Program for Social ICT Global Creative Leaders. Multi-method approaches for reducing problem drinking: Focusing on the application of psychological theories. (Individual Project; Principal Investigator). Grant amount: ¥300,000
	\item 2018---2019: The University of Tokyo, Graduate Program for Social ICT Global Creative Leaders. Development of a monitoring system using physiological measures for the prevention of psychiatric disorders. (Group Project; Principal Investigator). Grant amount: ¥650,000
	\item 2020---2024: Grant-in-Aid for Early-career Scientists, Japan Society for the Promotion of Science. Understanding emotional Difficulties in gaming disorder. (Individual Project; Principal Investigator). Grant amount: ¥3,700,000
	\item 2020---2023: Grant-in-Aid for JSPS Fellows (PD), Japan Society for the Promotion of Science. Comorbidities of and interventions for behavioral addiction among adolescents. (Individual Project; Principal Investigator). Grant amount: ¥4,160,000
	\item 2021---2024: Grant-in-Aid for Scientific Research (C), Japan Society for the Promotion of Science. Understanding the pathology and psychosocial risk factors of internet gaming disorder among children and youth. (Co-investigator).
	\item 2022---2023: Reiwa 4 nendo shougaisha sougou fukusi suishin jigyouhi hojyo kin, Ministry of Health, Labour and Welfare. Gyanburu tou izonshou mondai no jittai chousa no jisshi houhou no sakutei ni kakaru kentou. (Co-investigator).
	\item 2023---2027: Grant-in-Aid for Scientific Research (B), Japan Society for the Promotion of Science. Evaluating the effectiveness of cognitive behavioral therapy for problematic use of the internet and video gaming. (Principal investigator). Grant amount: ¥11,960,000
	\end{description}

\section{Society Affiliation}
\begin{itemize}
	\item Association of Japanese Clinical Psychology
%	\item Association for Psychological Science
	\item Japanese Medical Society of Alcohol and Addiction Studies
%	\item Japanese Psychological Association
	\item Japanese Society of Alcohol-Related Problems
%	\item Research Society on Alcoholism
%	\item Society for Personality and Social Psychology
\end{itemize}

\section{Ad Hoc Review}
\begin{itemize}
	\item Addictive Behaviors
	\item Addictive Behaviors Reports
	\item Brain and Behavior
	\item Cyberpsychology: Journal of Psychosocial Research on Cyberspace 
	\item INQUIRY
	\item International Journal of Human-Computer Interaction
	\item International Journal of Mental Health and Addiction
	\item Journal of Occupational Health
	\item Psychology Research and Behavior Management 
	\item Scientific Reports
\end{itemize}

\section{Award and Fellowship}
	\begin{description}
		\item 2017/04: \textbf{Doctoral Course (DC) Research Fellowship for Young Scientists}, Japan Society for the Promotion of Science
		\item 2017/09: \textbf{The 6th Kosugi Memorial Award (Distinguished Poster Presentation Award)}; Japanese Society of Alcohol-related Problems.
		\item 2020/04: \textbf{Postdoctoral (PD) Research Fellowship for Young Scientists}, Japan Society for the Promotion of Science
		\item 2021/03: \textbf{Distinguished Presence Award}, KDDI Research, Inc. (Co-investigator)
		\item 2022/10: \textbf{Young Scholar Outstanding Presentation Award (Oral presentation)}, The 22nd Annual Convention of Japanese Association for Cognitive Therapy
		\item 2022/04: \textbf{Overseas Research Fellowship}, Japan Society for the Promotion of Science
	\end{description}
	
\end{document}